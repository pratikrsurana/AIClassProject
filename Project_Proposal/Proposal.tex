\documentclass{article}
\title{Descriptive Project Title}
\author{First Author and Second Author and Third Author
%make sure to specify if the project is Programming or %Technology Driven
\thanks{CS256 Section 2 Fall 2017: Programming or Technology Driven Project}}
\date{\today}
\begin{document}

\maketitle

\begin{abstract}
Include a 250 word (maximum) abstract that addresses the following questions:
\begin{itemize}
\item What BIG question are you trying to answer?
\item Why is this question interesting/useful?
\item Which methods will you use to answer this question?
\item What data sets will you use?
\item What performance measure will you use to analyze performance?
\end{itemize}
\end{abstract}

\section{Introduction}

In one paragraph, explain the motivation for the question that you are trying to answer. 

In the second paragraph, describe any research or products that deal with similar questions or problems.  Make sure to cite all primary references, for example~\cite{einstein}.

In the third paragraph, explain how your project will differ from work described in the preceding paragraph. 

\section{Materials and Methods}
Explain in details how you will accomplish your goal.  Provide as many details on which tools will be used to collect the data; which programming language and modules will be used to build your agents.

Include a detailed description of your agent by specifying the following:
\subsection{Environment}
What will be the expected environment in which your agent will operate?
\subsection{Sensors}
How will the Agent sense the percepts from the Environment
\subsection{Percepts}
Include and describe sample percept sequences
\subsection{Actuators}
Describe which actuators will be used and what will be sent by the Agent back to the Environment.
\subsection{Performance Measures}
Describe at least two performance measures that you will use to analyze your agent.

\section{Experiments}
Describe in detail how you will test/evaluate your project.
If the project is technology-driven, it must be tested with real users to obtain performance measures.  E.g. if you are developing a mobile app for building custom-made online boot camps for technical skills, it must be tested with at least 10 different users to collect feedback on your new technology.

If you are doing programming-driven project, describe in details the data sets which will be used and make sure to provide citations to the original source, such as~\cite{knuthwebsite}. 

\bibliography{Proposal} 
\bibliographystyle{ieeetr}
\end{document}
